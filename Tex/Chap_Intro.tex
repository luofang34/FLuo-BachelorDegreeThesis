\chapter{引言}\label{chap:introduction}

\section{研究背景}
脑源性神经营养因子(Brain-derived neurotrophic factor, BDNF)是一个调控神经发育与突触功能的关键因子。然而,近来的研究发现脑源性神经营养因子同时也是一个调节哺乳动物能量平衡的关键因素\citep{xu2016neurotrophic}。一些全基因组关联分析(Genome-wide association study, GWAS)的结果显示, \textit{Bdnf}基因上的几处位点和肥胖发病有着高度的相关性\citep{thorleifsson2009genome,speliotes2010association,wen2012meta},\textit{Bdnf}基因及其一些受体发生突变的病例出现了严重的肥胖症\citep{gray2006hyperphagia}。而与此相符地, 在小鼠下丘脑中注射BDNF已被证实可以导致进食减少和能量消耗增加\citep{wang2007abrain,wang2007bbrain,wang2010brain,godar2011reduction},而敲除bdnf则会引发肥胖\citep{xu2003brain, unger2007selective, liao2012dendritically}。

在下丘脑中,BDNF在室旁核(PVN),腹内侧下丘脑(VMH)中的神经元中以高水平表达,并且在背内侧下丘脑(DMH)中以中等水平表达\citep{xu2003brain, unger2007selective, liao2012dendritically, an2015discrete}。将BDNF蛋白直接注射到VMH或PVN中会导致小鼠食物摄入的减少及能量消耗的增加\citep{wang2007abrain,wang2007bbrain,godar2011reduction}。从VMH和DMH中选择性敲除BDNF则会导致成年小鼠的食欲和食量的增加以及中度肥胖\citep{unger2007selective}。相反,从PVN中选择性敲除BDNF则会引起严重的肥胖,饮食过多,适应性热成像受损和运动活性降低\citep{an2015discrete}。详细的解剖学分析表明,前室旁核(anterior PVN)中的BDNF缺失会选择性地影响食物摄入,前室旁核(anterior PVN)选择性地影响棕色脂肪介导的产热(An等,2015)。 BDNF也在视前温敏神经元中表达(Tan等人,2016; Zhao等人,2017),并且这些神经元的激活诱导低温(Tan等人,2016)。然而,尚不清楚BDNF信号本身对于体温调节是否重要。