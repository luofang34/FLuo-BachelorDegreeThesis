\chapter{引言}\label{chap:introduction}

\section{研究背景}
脑源性神经营养因子(Brain-derived neurotrophic factor, BDNF)是一个调控神经发育与突触功能的关键因子。然而,近来的研究发现脑源性神经营养因子同时也是一个调节哺乳动物能量平衡的关键因素\citep{xu2016neurotrophic}。一些全基因组关联分析(Genome-wide association study, GWAS)的结果显示, \textit{Bdnf}基因上的几处位点和肥胖发病有着高度的相关性\citep{thorleifsson2009genome,speliotes2010association,wen2012meta},\textit{Bdnf}基因及其一些受体发生突变的病例出现了严重的肥胖症\citep{gray2006hyperphagia}。而与此相符地, 在小鼠下丘脑中注射BDNF已被证实可以导致进食减少和能量消耗增加\citep{wang2007abrain,wang2007bbrain,wang2010brain,godar2011reduction},而敲除\textit{bdnf}则会引发肥胖\citep{xu2003brain, unger2007selective, liao2012dendritically}。

在下丘脑中,BDNF在室旁核(PVN),腹内侧下丘脑(VMH)中的神经元中以高水平表达,并且在背内侧下丘脑(DMH)中以中等水平表达\citep{xu2003brain, unger2007selective,liao2012dendritically,an2015discrete}。将BDNF蛋白直接注射到VMH或PVN中会导致小鼠食物摄入的减少及能量消耗的增加\citep{wang2007abrain,wang2007bbrain,godar2011reduction}。从VMH和DMH中选择性敲除BDNF则会导致成年小鼠的食欲和食量的增加以及中度肥胖\citep{unger2007selective}。相反,从PVN中选择性敲除BDNF则会引起严重的肥胖,饮食过多,适应性热成像受损和运动活性降低\citep{an2015discrete}。详细的解剖学分析表明,前室旁核(anterior PVN)中的BDNF缺失会选择性地影响食物摄入,后室旁核(anterior PVN)中的缺失则会选择性地影响棕色脂肪介导的产热\citep{an2015discrete}。 BDNF也在视前温敏神经元中表达\citep{tan2016warm,zhao2017hypothalamic},而这些神经元的激活会导致失温\citep{tan2016warm}。然而,尚不清楚BDNF信号本身对于体温调节是否重要。

BDNF介导的能量稳态调节的神经回路尚不明确。 BDNF信号传导被认为是源自两种已知的调节能量稳态的信号通路\citep{waterson2015neuronal, krashes2016melanocortin},即弓状核(Arc)\citep{xu2003brain}和瘦素信号传导\citep{liao2012dendritically}的黑皮质素系统的下游。瘦素能够上调BDNF表达水平,并且注入BDNF可以复原由A$^{y}$突变引起的体重增加(A代表Agouti,是一种黑皮质素系统中的基因)。然而,BDNF神经元是否能作为下丘脑瘦素神经元、弓状刺豚鼠相关肽(AgRP)神经元或弓状前脑皮质素(POMC)神经元的突触后靶标仍然尚未可知。 PVN BDNF神经元(PVN$^{BDNF}$)直接投射到脊髓,它们可能在控制棕色脂肪的产热活性中发挥作用;而PVN$^{BDNF}$神经元在饮食调节中的突触靶标未知\citep{an2015discrete}。此外,PVN$^{BDNF}$神经元的全脑轴突投射模式目前尚未有所报道。因此,为了更深入地了解BDNF环路在能量稳态中的作用,识别影响、调节BDNF神经元的关键功能——如上文所述PVN和VMH中BDNF的功能——的上游输入和下游输出是至关重要的。系统地识别BDNF神经元上游突触前输入和下游突触后靶标神经元的全脑模式有助于描绘关键功能连接,为神经网络的后续研究提供帮助。因此,为了定位PVN$^{BDNF}$神经元的上游输入神经元的全脑分布,我们采用了基于改良狂犬病病毒的逆行跨单突触追踪技术,这种技术可以做到精确定位某种由遗传手段定义的细胞类型(Wickersham等。,2007)。此外,我们使用基于腺相关病毒(AAV)的荧光示踪剂分析了这些神经元的轴突投射模式。我们的研究结果显示PVNBDNF神经元接收来自数十个脑区的密集输入,其中包括已知的能量稳态调节中心,如VMH,弧和DMH,以及许多研究较少的能量静止区域,如侧隔膜(LS),视前区(POA),室旁丘脑核(PVA),透明带(ZI),下颌(S),外侧臂旁核(LPB),中缝大核(RMg)和中缝苍白球核(RPa)。几乎所有这些脑区域也接受来自PVNBDNF神经元的相互轴突投射。我们一起概述了PVNBDNF神经元介导的潜在神经电路,以供进一步研究。
