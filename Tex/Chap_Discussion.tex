\chapter{讨论}\label{chap:discussion}
在这项研究中,我们利用基于狂犬病病毒的逆向跨突触标记与基于腺相关病毒的顺行示踪方法系统地解析了PVN BDNF神经元的输入和输出。我们发现几个关键区域可能与PVN BDNF神经元在能量平衡控制中的作用有关,并且为后续研究概述了其大致的回路。

\section{标记BDNF神经元的技术}
有几种标记表达BDNF的神经元的方法,包括免疫组织化学法和遗传学方法。然而,由于缺乏针对BDNF的特异性抗体,研究人员倾向于例如捕获启动子/mRNA或标签蛋白的遗传学手段,此倾向是在最近推出可以在内源基因组环境中廉价且有效地编辑DNA序列的CRISPR/Cas9技术后\citep{ran2013genome, doudna2014new}变得格外明显。病毒内部核糖体进入位点(IRES)和自切割2A肽序列已广泛用于忠实地捕获目标mRNA和蛋白质,并揭示其内源表达模式\citep{vong2011leptin, daigle2018suite}。在这项研究中,为了忠实地标记BDNF神经元,我们使用CRISPR/Cas9基因组编辑技术在Bdnf终止密码子后敲入IRES-Cre序列,从而便Cre与Bdnf的mRNA一起转录,然后经由IRES调控的翻译过程介导进行翻译。接着,Cre重组酶将用于驱动含LoxP的遗传元件的表达。在通过将绿色荧光报告品系LSL.EGFPL10和BDNF-IRES-Cre品系小鼠杂交所得的F1代小鼠中,我们从几个已知的BDNF表达区域中发现了EGFP的表达模式,表明使用BDNF-IRES-Cre可以有效地标记BDNF神经元。

\section{跨突触标记与传统示踪}
\textit{Bdnf}基因是肥胖症患者中被发现突变最频繁的基因之一。近来的证据揭示了PVN BDNF神经元在调控进食和产热中的关键作用。然而,这些神经元的突触前输入之前是未知的,这限制了对BDNF参与的神经回路的理解。传统的逆向示踪技术,例如逆行蛋白霍乱毒素B(Cholera Toxin Subunit B,CTb)\citep{conte2009multiple}或是逆行示踪病毒比如逆行的AAV\citep{tervo2016designer}会无选择性地感染起始位点的神经元,因此不能组织特异地标记突触前输入。考虑到PVN神经元在解剖学和功能上的复杂性,这些传统方法很可能无法有效地揭示PVN BDNF神经元的信号输入。因此,我们采用了一个包含一种修饰过的RV和两个辅助AAV的三病毒体系,来使RV感染局限于BDNF神经元,并且只能逆向感染直接给予其输入的神经元。这个三病毒系统帮助解析了PVN BDNF神经元的全脑输入模式。PVN BDNF神经元展现出的输入模式和同区域其他种类的神经元明显不同,例如促肾上腺皮质素释放素受体神经元\citep{jiang2018local}。

\section{功能意义}
PVN BDNF神经元的上游突触前输入和下游轴突输出的具体定位揭示了几种可能在BDNF环路中起到控制进食及产热作用的解剖学连接。例如,我们确定了几个与PVN BDNF神经元有强烈相互连接的新脑区,包括LS,Amy,POA,ZI和S。 LS和Amy在情绪过程和压力反应中至关重要\citep{yadin1993role, menard1996lateral}。这些区域与PVN BDNF神经元之间的强烈和相互连接暗示LS/Amy-PVN$^{BDNF}$环路对于情绪,压力和能量平衡之间的相互作用可能存在一定重要性。有趣的是,LS也被证明可以为Arc AgRP和POMC神经元提供输入,这是调节摄食和能量消耗的两种关键神经元类型\citep{wang2015whole}。此外,向LS中注射阿片类药物或去甲肾上腺素在大鼠中具有促进食物的作用\citep{majeed1986stimulation,scopinho2008alpha1}。杏仁核是压力和情绪调节的中心,也是食物摄入的重要调节因子\citep{zhang2011amygdala,cai2014central,douglass2017central}。目前尚不清楚这些回路是否以及如何参与可能导致人类肥胖的情绪喂养\citep{kishi2005body}。需要进一步的研究来解决它们在调节摄食和产热方面的特定功能。

PVN BDNF神经元的上游突触前输入和下游轴突输出的具体定位揭示了几种可能在BDNF环路中起到控制进食及产热作用的解剖学连接。例如,我们确定了几个与PVN BDNF神经元强烈相互联系的新脑区,包括LS,Amy,POA,ZI和S. LS和Amy在情绪过程中至关重要和压力反应\citep{yadin1993role,menard1996lateral}。该



此外POA和ZI和PVN BDNF有相互的连接\figurename{\ref{fig:figure5}}。POA是一个已知的温度调节中枢,最近还被认为和体重和能量消耗的调节相关\citep{morrison2011central,yu2016glutamatergic,zhao2017hypothalamic,tan2018regulation}。

包括睡眠\citep{liu2017lhx6},觉醒\citep{roger1985afferents,power2001evidence},注意力\citep{chometton2017rostromedial,tait2017effects},运动\citep{milner1988electrical,supko1991activation,murer1993circling,perier2002behavioral},内脏活动\citep{huang1974differential,walsh1977some}和暴食\citep{zhang2017rapid}

It might be interesting to test whether this ZIePVHBDNF circuitry is involved in binge-like eating and obesity. 

总之,我们对PVN BDNF神经元的输入和输出的研究为将来的功能研究提供了指导。



