\chapter{讨论}\label{chap:discussion}
在这项研究中,我们利用基于狂犬病病毒的逆向跨突触标记与基于腺相关病毒的顺行示踪方法系统地解析了PVN BDNF神经元的输入和输出。

\section{标记BDNF神经元的技术}
有几种标记表达BDNF的神经元的方法,包括免疫组织化学法和遗传学方法。然而,由于缺乏针对BDNF的特异性抗体,研究人员倾向于例如捕获启动子/ mRNA或标签蛋白等遗传学手段,特别是在最近推出CRISPR / Cas9技术后(Ran)
等,2013; Doudna和Charpentier,2014)允许在内源基因组环境中廉价和有效地编辑DNA序列。病毒内部核糖体进入位点(IRES)和自切割2A肽序列已广泛用于忠实地捕获靶mRNA和蛋白质,并揭示其内源表达模式(Vong等,2011; Daigle等,2018) )。在这项研究中,为了忠实地标记BDNF神经元,我们使用CRISPR / Cas9基因组编辑在Bdnf终止密码子后立即敲入IRES-Cre序列,以便Cre将与Bdnf mRNA一起转录,然后通过IRES-翻译。介导的翻译。然后,Cre重组酶将用于驱动含LoxP的遗传元件的表达。在具有绿色荧光报告系列LSL.EGFPL10的BDNF-IRES-Cre小鼠杂交的F1中,我们在几个已知的BDNF表达区域中发现了EGFP的表达模式,表明通过使用BDNF-IRES-Cre标记BDNF神经元。驱动鼠标有效。

\section{跨突触标记与传统示踪}
\textit{Bdnf}基因是肥胖症患者中被发现突变最频繁的基因之一。近来的证据揭示了PVN BDNF神经元在调控进食和产热中的关键作用。然而,这些神经元的突触前输入之前是未知的,这限制了对BDNF参与的神经回路的理解。传统的逆向示踪技术,例如逆行蛋白霍乱毒素B(Cholera Toxin Subunit B,CTb)\citep{conte2009multiple}或是逆行示踪病毒比如逆行的AAV\citep{tervo2016designer}会无选择性地感染起始位点的神经元,因此不能组织特异地标记突触前输入。考虑到PVN神经元在解剖学和功能上的复杂性,这些传统方法很可能无法有效地揭示PVN BDNF神经元的信号输入。因此,我们采用了一个包含一种修饰过的RV和两个辅助AAV的三病毒体系,来使RV感染局限于BDNF神经元,并且只能逆向感染直接给予其输入的神经元。这个三病毒系统帮助解析了PVN BDNF神经元的全脑输入模式。PVN BDNF神经元展现出的输入模式和同区域其他种类的神经元明显不同,例如促肾上腺皮质素释放素受体神经元\citep{jiang2018local}。

\section{功能意义}


此外POA和ZI和PVN BDNF有相互的连接\figurename{\ref{fig:figure5}}。POA是一个已知的温度调节中枢


In addition, the POA and the ZI are mutually connected with  neurons (Fig. 5). The POA is a known center for body tem- perature regulation and has also recently been suggested in regu- lating body weight and energy expenditure (Shaun F. Morrison, 2011; Yu et al., 2016; Zhao et al., 2017; Tan and Knight, 2018). It might be interesting to test whether this POAePVHBDNF circuitry is involved in temperature-dependent feeding control and body weight regulation. The ZI is one of the functionally ‘uncertain’ re- gions in the brain, which has been associated with diverse func- tions including sleep (Liu et al., 2017), arousal (Roger and Cadusseau, 1985; Power et al., 2001), attention (Chometton et al.,
38 F. Luo et al. / Journal of Genetics and Genomics 46 (2019) 31e40
2017; Tait et al., 2017), locomotion (Milner and Mogenson, 1988; Supko et al., 1991; Murer and Pazo, 1993; Perier et al., 2002), visceral activity (Huang and Mogenson, 1974; Walsh et al., 1977) and binge-like eating (Zhang and van den Pol, 2017). It might be interesting to test whether this ZIePVHBDNF circuitry is involved in binge-like eating and obesity. In all, our mapping of the inputs and outputs of PVN BDNF neurons provides a guideline for further functional studies.


