%---------------------------------------------------------------------------%
%->> 封面信息及生成
%---------------------------------------------------------------------------%
%-
%-> 中文封面信息
%-
\confidential{}% 密级:只有涉密论文才填写
\schoollogo{scale=0.5}{ShanghaiTech_logo}% 校徽
\title{test}% 论文中文题目
\author{罗\quad 放}% 论文作者
\advisor{\quad\quad\quad\quad 沈\quad 伟 \quad\quad 上海科技大学生命科学与技术学院}% 指导教师:姓名 专业技术职务 工作单位

\advisorsec{}% 第二指导老师:按情况填写
\degree{学士}% 学位:学士、硕士、博士
\degreetype{理学}% 学位类别:理学、工学、工程、医学等
\major{生命科学}% 二级学科专业名称
\institute{生命科学与技术学院}% 院系名称
\chinesedate{2019~年~6~月}% 毕业日期:夏季为6月、冬季为12月
%-
%-> 英文封面信息
%-

\englishtitle{Whole-brain patterns of the presynaptic inputs and axonal projections \hspace{+1.0em} of \newline BDNF neurons in the paraventricular nucleus}% 论文英文题目
\englishauthor{Fang Luo}% 论文作者
\englishadvisor{Supervisor: Wei L. Shen}% 指导教师
\englishdegree{Bachelor}% 学位:Bachelor, Master, Doctor。封面格式将根据英文学位名称自动切换,请确保拼写准确无误
\englishdegreetype{Science}% 学位类别:Philosophy, Natural Science, Engineering, Economics, Agriculture 等
\englishthesistype{thesis}% 论文类型: thesis, dissertation
\englishmajor{Life Science}% 二级学科专业名称
\englishinstitute{School of Life Science and Technology, ShanghaiTech University}% 院系名称
\englishdate{June, 2019}% 毕业日期:夏季为June、冬季为December

%-
%-> 生成封面
%-
\maketitle% 生成中文封面
\makeenglishtitle% 生成英文封面
%-
%-> 作者声明
%-
\makedeclaration% 生成声明页
%-
%-> 中文摘要
%-
\chapter*{摘\quad 要}\chaptermark{摘\quad 要}% 摘要标题
\setcounter{page}{1}% 开始页码
\pagenumbering{Roman}% 页码符号

脑源性神经营养因子(BDNF)在人类肥胖中起着至关重要的作用。然而,支持BDNF介导的能量稳态控制的神经回路仍然很大程度上未被揭露。为了绘制可能为下丘脑室旁核(PVN)的BDNF神经元------一种调节摄食和产热的关键类型细胞,提供传入信号或接收该区域传入信号的关键区域,我们使用基于狂犬病病毒的跨突触标记和基于腺相关病毒的顺行示踪技术揭示它们的全脑分布。我们发现数十个脑区为PVN BDNF神经元提供密集输入或接收密集输入,包括几个已知的体重控制区域和几个可能对BDNF介导的能量稳态调节具有重要功能的新区域。有趣的是,几个区域与PVN BDNF神经元显示非常密集的相互连接,包括侧间隔,视前区,腹内侧下丘脑核,室旁丘脑核,透明带,侧臂旁核,下颌,中缝大核,和中缝苍白球核。这些比较强的解剖学连接可能暗示着重要的功能相关性。因此,我们提供了由PVN BDNF神经元介导的潜在神经回路的概要,从而有助于解决复杂的肥胖相关神经网络。

\keywords{脑源性神经营养因子,跨突触示踪,肥胖,进食,产热}% 中文关键词
%-
%-> 英文摘要
%-
\chapter*{Abstract}\chaptermark{Abstract}% 摘要标题
Brain-derived neurotrophic factor (BDNF) plays a crucial role in human obesity. Yet, the neural circuitry supporting the BDNF-mediated control of energy homeostasis remains largely undefined. To map key regions that might provide inputs to or receive inputs from the paraventricular nucleus (PVN) BDNF neurons, a key type of cells in regulating feeding and thermogenesis, we used rabies virus-based transsynaptic labeling and adeno-associated virus based anterograde tracing techniques to reveal their whole-brain distributions. We found that dozens of brain regions provide dense inputs to or receive dense inputs from PVN BDNF neurons, including several known weight control regions and several novel regions that might be functionally important for the BDNF-mediated regulation of energy homeostasis. Interestingly, several regions show very dense reciprocal connections with PVN BDNF neurons, including the lateral septum, the preoptic area, the ventromedial hypothalamic nucleus, the paraventricular thalamic nucleus, the zona incerta, the lateral parabrachial nucleus, the subiculum, the raphe magnus nucleus, and the raphe pallidus nucleus. These strong anatomical connections might be indicative of important functional connections. Therefore, we provide an outline of potential neural circuitry mediated by PVN BDNF neurons, which might be helpful to resolve the complex obesity network.

\englishkeywords{BDNF, Transsynaptic tracing, Obesity,
Feeding, Thermogenesis}% 英文关键词
%---------------------------------------------------------------------------%
