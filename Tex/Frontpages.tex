%---------------------------------------------------------------------------%
%->> 封面信息及生成
%---------------------------------------------------------------------------%
%-
%-> 中文封面信息
%-
\confidential{}% 密级:只有涉密论文才填写
\schoollogo{scale=0.5}{ShanghaiTech_logo}% 校徽
\title{上海科技大学学位论文\LaTeX{}模板 {$~^{\pi}\pi^{\pi}$}}% 论文中文题目
\author{曾勃}% 论文作者
\advisor{曾勃~教授~上海科技大学勃学研究所}% 指导教师:姓名 专业技术职务 工作单位
\advisorsec{}% 第二指导老师:按情况填写
\degree{学士}% 学位:学士、硕士、博士
\degreetype{哲学}% 学位类别:理学、工学、工程、医学等
\major{勃学}% 二级学科专业名称
\institute{上海科技大学勃学研究所}% 院系名称
\chinesedate{2018~年~6~月}% 毕业日期:夏季为6月、冬季为12月
%-
%-> 英文封面信息
%-

\englishtitle{of PVN BDNF neurons}% 论文英文题目
\englishauthor{Fang Luo}% 论文作者
\englishadvisor{Supervisor: Wei L. Shen}% 指导教师
\englishdegree{Bachelor}% 学位:Bachelor, Master, Doctor。封面格式将根据英文学位名称自动切换,请确保拼写准确无误
\englishdegreetype{Science}% 学位类别:Philosophy, Natural Science, Engineering, Economics, Agriculture 等
\englishthesistype{thesis}% 论文类型: thesis, dissertation
\englishmajor{Life Science}% 二级学科专业名称
\englishinstitute{School of Life Science and Technology, ShanghaiTech University}% 院系名称
\englishdate{June, 2019}% 毕业日期:夏季为June、冬季为December
%-
%-> 生成封面
%-
\maketitle% 生成中文封面
\makeenglishtitle% 生成英文封面
%-
%-> 作者声明
%-
\makedeclaration% 生成声明页
%-
%-> 中文摘要
%-
\chapter*{摘\quad 要}\chaptermark{摘\quad 要}% 摘要标题
\setcounter{page}{1}% 开始页码
\pagenumbering{Roman}% 页码符号

本文是中国科学院大学学位论文模板ucasthesis的使用说明文档。主要内容为介绍\LaTeX{}文档类ucasthesis的用法,以及如何使用\LaTeX{}快速高效地撰写学位论文。

\keywords{中国科学院大学,学位论文,\LaTeX{}模板}% 中文关键词
%-
%-> 英文摘要
%-
\chapter*{Abstract}\chaptermark{Abstract}% 摘要标题
Brain-derived neurotrophic factor (BDNF) plays a crucial role in human obesity. Yet, the neural circuitry supporting the BDNF-mediated control of energy homeostasis remains largely undefined. To map key regions that might provide inputs to or receive inputs from the paraventricular nucleus (PVN) BDNF neurons, a key type of cells in regulating feeding and thermogenesis, we used rabies virus-based transsynaptic labeling and adeno-associated virus based anterograde tracing techniques to reveal their whole-brain distributions. We found that dozens of brain regions provide dense inputs to or receive dense inputs from PVN BDNF neurons, including several known weight control regions and several novel regions that might be functionally important for the BDNF-mediated regulation of energy homeostasis. Interestingly, several regions show very dense reciprocal connections with PVN BDNF neurons, including the lateral septum, the preoptic area, the ventromedial hypothalamic nucleus, the paraventricular thalamic nucleus, the zona incerta, the lateral parabrachial nucleus, the subiculum, the raphe magnus nucleus, and the raphe pallidus nucleus. These strong anatomical connections might be indicative of important functional connections. Therefore, we provide an outline of potential neural circuitry mediated by PVN BDNF neurons, which might be helpful to resolve the complex obesity network.

\englishkeywords{BDNF, Transsynaptic tracing, Obesity,
Feeding, Thermogenesis}% 英文关键词
%---------------------------------------------------------------------------%
