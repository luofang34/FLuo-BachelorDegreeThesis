%\chaptermark{版权声明}

\chapter{版权声明 \\ Copyright notice}


本文之全部或部分内容于之前曾公开发表。这项之前发表的工作可由\citep{luo2019whole}或\href{https://doi.org/10.1016/j.jgg.2018.11.004}{doi:10.1016/j.jgg.2018.11.004}检索到。

This thesis, in part or in whole contains contents from a previously published work. Such work could be retrieved by \citep{luo2019whole} or \href{https://doi.org/10.1016/j.jgg.2018.11.004}{doi:10.1016/j.jgg.2018.11.004}.


\section*{} %请根据实际更改增删期刊出版协议等内容
已发表工作的版权可能已依协议转让予期刊和/或其出版商。然而,依据\href{http://www.jgenetgenomics.org/fileup/JGG_publishing_agreement.pdf}{期刊出版协议}与出版商的\href{https://www.elsevier.com/journals/journal-of-genetics-and-genomics/16738527/guide-for-authors}{作者指引},个人使用和于机构内部非商业使用这些工作的权利(例如这篇学位论文)依然由作者保留。因此,作者有权签署上海科技大学学位论文授权使用声明并提供其中所列出的授权。然而,此授权并不意味着对商业使用的许可。

Copyright of the previously published works might already transfered to the journals and/or publishers. However, rights for non-commerical personal use and internal institutional Use (such as reuse in this thesis) is retained by the author accroding to the \href{http://www.jgenetgenomics.org/fileup/JGG_publishing_agreement.pdf}{Journal Publishing Agreement} and the publisher's \href{https://www.elsevier.com/journals/journal-of-genetics-and-genomics/16738527/guide-for-authors}{guideline}. Therefore, the author has the preoper rights to sign and to authorize the \textit{上海科技大学学位论文授权使用声明}(ShanghaiTech University degree thesis authorized use statement). However, such authorization does not include the permission for commercial use.


%\section*{字符}
%\nomenclatureitem[\textbf{Unit}]{\textbf{Symbol}}{\textbf{Description}}
%\nomenclatureitem[$\Unit{m^{2} \cdot s^{-2} \cdot K^{-1}}$]{$R$}{the gas constant}
%\nomenclatureitem[$\Unit{m^{2} \cdot s^{-2} \cdot K^{-1}}$]{$C_v$}{specific heat capacity at constant volume}
%\nomenclatureitem[$\Unit{m^{2} \cdot s^{-2} \cdot K^{-1}}$]{$C_p$}{specific heat capacity at constant pressure}
%\nomenclatureitem[$\Unit{m^{2} \cdot s^{-2}}$]{$E$}{specific total energy}
%\nomenclatureitem[$\Unit{m^{2} \cdot s^{-2}}$]{$e$}{specific internal energy}
%\nomenclatureitem[$\Unit{m^{2} \cdot s^{-2}}$]{$h_T$}{specific total enthalpy}
%\nomenclatureitem[$\Unit{m^{2} \cdot s^{-2}}$]{$h$}{specific enthalpy}
%\nomenclatureitem[$\Unit{kg \cdot m \cdot s^{-3} \cdot K^{-1}}$]{$k$}{thermal conductivity}
%\nomenclatureitem[$\Unit{kg \cdot m^{-1} \cdot s^{-2}}$]{$S_{ij}$}{deviatoric stress tensor}
%\nomenclatureitem[$\Unit{kg \cdot m^{-1} \cdot s^{-2}}$]{$\tau_{ij}$}{viscous stress tensor}
%\nomenclatureitem[$\Unit{1}$]{$\delta_{ij}$}{Kronecker tensor}
%\nomenclatureitem[$\Unit{1}$]{$I_{ij}$}{identity tensor}

%\section*{算子}
%\nomenclatureitem{\textbf{Symbol}}{\textbf{Description}}
%\nomenclatureitem{$\Delta$}{difference}
%\nomenclatureitem{$\nabla$}{gradient operator}
%\nomenclatureitem{$\delta^{\pm}$}{upwind-biased interpolation scheme}


\chapter{名词缩写}
\chaptermark{名词缩写}
\nomenclatureitem{3V}{3rd ventricle}
\nomenclatureitem{3V}{3rd ventricle}
\nomenclatureitem{4N}{trochlear nucleus}
\nomenclatureitem{4V}{4th ventricle}
\nomenclatureitem{Acb}{accumbens nucleus}
\nomenclatureitem{AH}{anterior hypothalamic area}
\nomenclatureitem{AHiPM}{amygdalo- hippocampal area, posteromedial part}
\nomenclatureitem{AHP}{anterior hypothalamic area, posterior part}
\nomenclatureitem{Amy}{amygdala}
\nomenclatureitem{Amy}{amygdaloid}
\nomenclatureitem{Ant}{anterior lobe cerebellum}
\nomenclatureitem{AOM}{anterior olfactory nucleus, medial part}
\nomenclatureitem{APir}{amygdalopiriform transition area}
\nomenclatureitem{Aq}{aqueduct}
\nomenclatureitem{Arc}{arcuate hypothalamic nucleus}
\nomenclatureitem{BLP}{basolateral amygdaloid nucleus, posterior part}
\nomenclatureitem{BST}{bed nucleus of the stria terminalis}
\nomenclatureitem{Cg}{cingulate cortex}
\nomenclatureitem{Cir}{circular nucleus}
\nomenclatureitem{CnF}{cuneiform nucleus}
\nomenclatureitem{D3V}{dorsal 3rd ventricle}
\nomenclatureitem{DM}{dorsomedial hypothalamic nucleus}
\nomenclatureitem{DMH}{dorsomedial hypothalamic nucleus}
\nomenclatureitem{DP}{dorsal peduncular cortex}
\nomenclatureitem{DR}{dorsal raphe nucleus}
\nomenclatureitem{DTM}{dorsal tuberomammillary nucleus}
\nomenclatureitem{DTT}{dorsal tenia tecta}
\nomenclatureitem{Ent}{entorhinal cortex}
\nomenclatureitem{f}{fornix}
\nomenclatureitem{fr}{fasciculus retroflexus}
\nomenclatureitem{Fu}{bed nucleus of stria terminalis, fusiform part}
\nomenclatureitem{gcc}{genu of the corpus callosum}
\nomenclatureitem{Gi}{gigantocellular reticular nucleus}
\nomenclatureitem{IAD}{interanterodorsal thalamic}
\nomenclatureitem{ICjM}{islands of Calleja, major island}
\nomenclatureitem{IL}{infralimbic cortex}
\nomenclatureitem{IOM}{inferior olive, medial nucleus}
\nomenclatureitem{IOPr}{inferior olive, principal nucleus}
\nomenclatureitem{LA}{lateroanterior hypothalamic nucleus}
\nomenclatureitem{LDTg}{laterodorsal tegmental nucleus}
\nomenclatureitem{LH}{lateral hypothalamic area}
\nomenclatureitem{LMol}{lacunosum moleculare layer of the hippocampus}
\nomenclatureitem{LO}{lateral orbital cortex}
\nomenclatureitem{LPB}{lateral parabrachial nucleus}
\nomenclatureitem{LPGi}{lateral paragigantocellular nucleus}
\nomenclatureitem{LPO}{lateral preoptic area}
\nomenclatureitem{LS}{lateral septal nucleus}
\nomenclatureitem{LV}{lateral ventricle}
\nomenclatureitem{M2}{secondary motor cortex}
\nomenclatureitem{Me}{medial amygdaloid nucleus}
\nomenclatureitem{ml}{medial lemniscus}
\nomenclatureitem{MnPO}{median preoptic nucleus}
\nomenclatureitem{MO}{medial orbital cortex}
\nomenclatureitem{MPA}{medial preoptic area}
\nomenclatureitem{MPO}{medial preoptic nucleus}
\nomenclatureitem{MS}{medial septal nucleus}
\nomenclatureitem{opt}{optic tract}
\nomenclatureitem{OV}{olfactory ventricle}
\nomenclatureitem{P5}{peritrigeminal zone}
\nomenclatureitem{PAG}{periaqueductal gray}
\nomenclatureitem{PF}{parafascicular thalamic nucleus}
\nomenclatureitem{PH}{posterior hypothalamic area}
\nomenclatureitem{PMCo}{basolateral amygdaloid nucleus, posterior part}
\nomenclatureitem{PMV}{premammillary nucleus, ventral part}
\nomenclatureitem{PnV}{pontine reticular nucleus, ventral part}
\nomenclatureitem{PP}{peripeduncular nucleus}
\nomenclatureitem{PrL}{prelimbic cortex}
\nomenclatureitem{PS}{parastrial nucleus}
\nomenclatureitem{pv}{periventricular fiber system}
\nomenclatureitem{PVA}{paraventricular thalamic nucleus, anterior part}
\nomenclatureitem{PVN}{paraventricular hypothalamic nucleus}
\nomenclatureitem{py}{pyramidal cell layer of the hippocampus}
\nomenclatureitem{RC}{raphe cap}
\nomenclatureitem{RCh}{retrochiasmatic area}
\nomenclatureitem{Re}{reuniens thalamic nucleus}
\nomenclatureitem{RMg}{raphe magnus nucleus}
\nomenclatureitem{RPa}{raphe pallidus nucleus}
\nomenclatureitem{Rt}{reticular thalamic nucleus}
\nomenclatureitem{RtTg}{reticulotegmental nucleus of the pons}
\nomenclatureitem{RVL}{rostroventrolateral reticular nucleus}
\nomenclatureitem{S}{subiculum}
\nomenclatureitem{scp}{superior cerebellar peduncle}
\nomenclatureitem{SFO}{subfornical organ}
\nomenclatureitem{SHi}{septohippocampal nucleus}
\nomenclatureitem{sm}{stria medullaris of the thalamus}
\nomenclatureitem{SO}{supraoptic nucleus}
\nomenclatureitem{SPa}{subparaventricular zone of the hypothalamus}
\nomenclatureitem{SPF}{subparafascicular thalamic nucleus}
\nomenclatureitem{VMH}{ventromedial hypothalamic nucleus}
\nomenclatureitem{VMPO}{ventromedial preoptic nucleus}
\nomenclatureitem{VO}{ventral orbital cortex}
\nomenclatureitem{VOLT}{vascular organ of the lamina terminalis}
\nomenclatureitem{vsc}{ventral spinocerebellar tract}
\nomenclatureitem{Xi}{xiphoid thalamic nucleus}
\nomenclatureitem{ZI}{zona incerta}