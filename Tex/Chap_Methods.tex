\chapter{Materials and methods}

\section{小鼠}
动物管理和使用符合上海科技大学,上海上海南方模式生物科技发展有限公司和中国政府法规的指导。所有实验均在控制温度(22-25$^\circ$C)下饲养的雄性小鼠(8e16周龄)进行,12小时光照/黑暗循环(黑暗时间:上午8点至晚上8点),无限制饮水和食物(4%脂肪SPF级啮齿动物饲料)。
Cre依赖的报告小鼠EEF1A1-LSL.EGFPL10(L10是一种核糖体驻留蛋白)由最初在(\cite{stanley2013profiling})提出该品系的Jeff Friedman博士赠送。小鼠品系C57BL/6J(#000664)来自Jackson Laboratory(USA),购自Silaike Experiment Animal Co.,Ltd。(中国上海)。BDNF-IRES-Cre小鼠由北京协和医学院的核心设施(??)靶向CRISPR/Cas9基因组制备。简而言之,通过CRISPR编码外显子将IRES-Cre序列插入BDNF编码外显子的3'UTR,得到Mus musculus C57BL/6J-BDNF-IRES-Cre。选择位于BDNF编码外显子上的靶位点1(5'-GGGGCATAGACAAAAGGCACTGG-3')和位于其3'UTR上的靶位点2(5'-GGACA TATCCATGACCTGAAAGG-30)用于在Bdnf终止密码子之后整合IRES-Cre序列。它们相应的指导序列M-Bdnf-EA-g-up(50-caccCTGGAGAACTACTGCAACTA-30)和M-Bdnf-EA-g-down(50-aaacTAGTTGCAGTAGTTCTCCAG-30)或M-Bdnf-EB-g-up(50-将TAGGA CATATCCATGACCTGAA-30)和M-Bdnf-EB-g-down(50 -AAACTT CAGGTCATGGATATGT-30)退火并如所述(\cite{mizuno2014simple})插入pX330的进入位点并命名为pX330-Bdnf。 -EA,pX330-Bdnf-EB。
\section{}

\section{}

\section{组织学与免疫组化检验}

\section{成像与数据分析}
鼠脑切片标本的图片使用了虚拟载玻片显微镜(Virtual Slide VS120, Olympus, Japan)和转盘式扫描激光共聚焦显微镜(Cell Observer SD, Zeiss, Germany)进行成像。每个核团的细胞数目使用ImageJ软件(NIH)进行了人工统计,取样了每只小鼠1/3的脑切片。脑区定位参照了小鼠全脑图谱(George Paxinos \& Keith Franklin, 2ed.)。