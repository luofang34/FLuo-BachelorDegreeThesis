\chapter{材料和方法}

\section{小鼠}
动物管理和使用符合上海科技大学,上海上海南方模式生物科技发展有限公司和中国政府法规的指导。所有实验均在控制温度(22-25$^\circ$C)下饲养的雄性小鼠(8-16周龄)进行,动物饲养于12小时光照/黑暗循环(黑暗时间:上午8点至晚上8点),无限制饮水和食物(4%脂肪SPF级啮齿动物饲料)。
Cre依赖的报告小鼠EEF1A1-LSL.EGFPL10(L10是一种核糖体驻留蛋白)\citep{stanley2013profiling}由建立品系的Jeff Friedman博士赠予。小鼠品系C57BL/6J(#000664, Jackson Laboratory(USA),购自上海斯莱克实验动物有限公司)。BDNF-IRES-Cre小鼠由北京协和医学院利用CRISPR/Cas9靶向制备。简而言之,\textit{Mus musculus} C57BL/6J-BDNF-IRES-Cre是通过将IRES-Cre序列插入BDNF编码外显子的3'UTR,得到。选择位于BDNF编码外显子上的靶位点1(5'-GGGGCATAGACAAAAGGCACTGG-3')和位于其3'UTR上的靶位点2(5'-GGACATATCCATGACCTGAAAGG-30)用于在\textit{Bdnf}终止密码子之后整合IRES-Cre序列。它们相应的引导序列M-Bdnf-EA-g-up(5'-caccCTGGAGAACTACTGCAACTA-3')和M-Bdnf-EA-g-down(5'-aaacTAGTTGCAGTAGTTCTCCAG-3')或M-Bdnf-EB-g-up(5'-TAGGACATATCC\\ATGACCTGAA-3')和M-Bdnf-EB-g-down(5' -AAACTTCAGGTCATGGATATGT-3'),按\citep{mizuno2014simple}所述退火并插入到pX330的插入位点,分别命名为\textit{pX330-Bdnf-EA},\textit{pX330-Bdnf-EB}。pBDNF/IRES-Cre供体质粒的两臂上含有\textit{Bdnf} gDNA 和一段 IRES-Cre序列。其5'臂上,为\textit{Bdnf}终止密码子前一段770-bp的序列,其3'臂上为\textit{Bdnf}终止密码子后一段8000-bp的序列。两臂之间,插入了一段IRES-Cre编码序列。为了防止被sgRNA误认为目标,以下几个位点进行了不改变氨基酸残基的同义突变:GGGGCATAGACAAA(G)AGG(CGA), ggacatatccatgacctgaaagg(C)。pBDNF/IRES-Cre, pX330-Bdnf-EA, 和 pX330-Bdnf-EB 被注入受精的卵母细胞中,存活的胚胎被移植到假孕的雌性实验动物体内。产生的小鼠使用鼠尾DNA扩增鉴定,使用LA Taq HS DNA 聚合酶 (RR042, Takara, Japan),上游引物(5'-GAAAGTTCTAACCTGTTCTGTGTCTGT-3'),下游引物 (5'-GCAATTTTCCCTGACCCAT-3')。

\section{病毒载体构建}
所有病毒均由武汉枢密脑科学技术有限公司(BrainVTA Co., Ltd., Wuhan, China)包装并在使用前保持冻结。狂犬病病毒SAD19-$\Delta$G-mCherry(EnvA)滴度为$2 \times 10^8 \text{IFU}/\mu L$, 搭载RG和TVA序列的Cre依赖2/9型AAV载体滴度为约$2 \times 10^12 \text{拷贝}/\mu L$。 所有病毒载体的构建方式均参照之前的文献\citep{zhang2017whole}。

\section{立体定位病毒注射}
在所有立体定位病毒注射过程中,小鼠均由经呼吸面具持续供给的异氟烷麻醉,并固定于一台立体定位设备(Model 942, David Kopf Instruments, USA)。两种辅助病毒AAV-DIO-EGFP-2A-TVA和AAV-DIO-RG以$1:1$比例混合后,$150nL$的混合物被双侧注射到目标脑区(PVN: bregma -0.5 mm, midline $\pm$ 0.25 mm, skull surface -4.9 mm)。在注射$100 nL$ SAD19-$\Delta$G-mCherry(EnvA)前,小鼠被给予三周恢复时间并使AAV辅助病毒得到表达。进行立体定位注射时,使用了一个加热垫来维持小鼠体温。在狂犬病病毒注射一周后,实验动物被处死。

\section{组织学与免疫组化检验}
小鼠由经呼吸面具持续供给的异氟烷麻醉后,经心灌流PBS缓冲液,随后灌流含4\%多聚甲醛(paraformaldehyde, PFA)的PBS溶液。鼠脑被分离并后固定3h后,使用30\%蔗糖溶液在4$^\circ$C下处理2天。鼠脑用冰冻包埋剂(Leica Microsystems, Germany)包覆并用冷冻切片机(CM3050S, Leica Microsystems)切为$40\mu m$厚切片,并贴片在黏附载玻片(80304, Cytoglas, China)上。进行荧光免疫染色时,脑片使用PBS浸洗3次,每次10min。随后,使用含2.5\% (v/v) 正常山羊血清,1.5\% (w/v) 牛血清白蛋白, 0.1\% Triton\texttrademark{} X-100 (v/v)和0.05\% NaN$_3$ (w/v) 于PBS的封闭缓冲液处理30min。随后,脑片与鸡抗GFP初级抗体(ab13970, Abcam, UK)在4$^\circ$C下孵育过夜。之后,脑片用PBST缓冲液(含0.1\% v/v Triton\texttrademark{} X-100 的PBS溶液)浸洗3次,之后与偶联DyLight 488的山羊抗鸡抗体(SA5-10070, Invitrogen, USA)在室温下孵育6h。上述所有抗体均以1:500稀释于封闭缓冲液中。随后,脑片用PBS缓冲液浸洗3次并用DAPI-Fluoromount-G荧光封片剂(0100-20, Southern Biotech, China)封片。

\section{成像与数据分析}
鼠脑切片标本的图片使用了虚拟载玻片显微镜(Virtual Slide VS120, Olympus, Japan)和转盘式扫描激光共聚焦显微镜(Cell Observer SD, Zeiss, Germany)进行成像。每个核团的细胞数目使用ImageJ软件(NIH)进行了人工统计,统计取样了每只小鼠1/3的脑切片。脑区定位参照了小鼠全脑图谱(George Paxinos \& Keith Franklin, 2ed.)。
