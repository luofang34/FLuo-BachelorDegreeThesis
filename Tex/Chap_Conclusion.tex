\chapter{总结}
在这项研究中,我们利用基于狂犬病病毒的逆向跨突触标记与基于腺相关病毒的顺行示踪方法系统地解析了PVN BDNF神经元的输入和输出。这项研究使用BDNF-IRES-Cre小鼠与三病毒标记体系,类型特异地选择了PVN BDNF神经元作为起始细胞,并由此获取到了PVN BDNF神经元在全脑的投射形态与PVN BDNF神经元的直接输入神经元的分布形态。

实验结果显示,给予PVN BDNF神经元直接输入的脑区包括VO, LO, DTT, LS, POA, Rt, SO, MeA, Sub, PVA, PH, VMH, Arc, ZI, PMV, BLp, py, PAG, S, RMg和RPa,PVN BDNF神经元接收来自全脑26423±5500个(n=3只小鼠)神经元的传入信号,大多数传入神经元(约52%)来自下丘脑,丘脑占约4%,中隔占约10%,中脑占约5%,脑干占约4%。PVN BDNF神经元的投射密度最大的部位位于下丘脑,如POA, DMH, VMH和ZI。 我们还在丘脑的PVA,中脑的PAG,隔膜的Amy和LS中发现了密集的投射。同时,我们在脑干中发现了显着的投射,例如LPB,RMg和RPa。我们发现,一些脑区与PVN BDNF神经元有非常密集的相互投射,这些脑区包括LS, POA, VMH, PVA, ZI, LPB, S, RMg和RPa。

