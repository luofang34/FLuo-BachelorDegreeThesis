\chapter{结果}\label{chap:results}

\section{Labeling BDNF neurons by using BDNF-IRES-Cre driver mice}
我们使用Cre/LoxP的方法来标记表达BDNF的神经元,通过Cre重组酶来驱动含有LoxP位点的荧光报告基因的表达。Bdnf基因具有复杂的结构,通过9个上游启动子来驱动一个小外显子被剪接到含有BDNF的完整编码区的共同外显子上(???)(\cite{timmusk1993multiple}, \cite{altieri2004expression})从而使9种不同的转录物编码相同的蛋白质,并且不同的转录物可使BDNF蛋白质在不同区域或发育阶段的不同细胞类型中表达以实现不同的生理功能(\cite{sakata2009critical}, \cite{maynard2016functional})因此,为了驱动Cre在所有表达BDNF的神经元中表达,我们使用CRISPR/Cas9基因编辑技术在共同外显子的终止密码子后敲入IRES-Cre(\figurename{1A})。在基因组重组序列验证成功后,我们将BDNF-IRES-Cre小鼠与绿色荧光报告系LSL.EGFPL10(\cite{stanley2013profiling})杂交,并在许多已知的下丘脑脑区,包括POA(\cite{tan2016warm}, \cite{zhao2017hypothalamic}),PVN(\cite{an2015discrete}),DMH和VMH(\cite{unger2007selective,} \cite{wang2010brain})中验证GFP的表达(\figurename{1B})。

\section{Viewing presynaptic inputs to PVNBDNF neurons by using rabies virus}
我们将三病毒系统与Cre/LoxP基因表达系统(\cite{wickersham2007monosynaptic})相结合,从而实现细胞类型特异性逆行跨单突触追踪(\figurename{1C})糖蛋白(RG)缺陷的SAD19-DG-mCherry(EnvA)狂犬病用禽肉瘤-白血病病毒包膜蛋白(EnvA)假型化(?????)。这使得病毒能够选择性地感染表达TVA的哺乳动物神经元,而TVA则是EnvA的同源病毒受体。将两种Cre依赖性AAV辅助病毒(AAV-DIO-EGFP-2A-TVA和AAV-DIO-RG)立体定位注射到BDNF-IRES-Cre驱动小鼠的双侧PVN中(\figurename{1C})。这些辅助病毒能表达EGFP-TVA从而使得狂犬病病毒可以感染胞体,并使得狂犬病RG可以从初始被感染神经元跨突触扩散。在3周后注射狂犬病病毒SAD19-DG-mCherry(EnvA)到相同区域以启动EGFP-TVA和RG的表达。

\section{Input patterns of PVNBDNF neurons}


\section{Axonal projection patterns of PVNBDNF neurons}

\section{Reciprocal projections between PVN BDNF neurons and their presynaptic inputs}